\documentclass[12pt, a4paper, twoside]{report}
\usepackage[T1,T2A]{fontenc}
\usepackage[utf8]{inputenc}
\usepackage[english,bulgarian,ukrainian,russian]{babel}
\usepackage{graphicx}
\usepackage{array}
\usepackage{tabularx}
\usepackage{listings}
\usepackage{color}
\usepackage{minted}
\usepackage{float}
\usepackage{enumitem}
\setlist{nolistsep, itemsep=0.3cm,parsep=0pt}
\usepackage{geometry}
\usepackage{csquotes}
\usepackage{soul}

\begin{document}

\begin{center}
MINISTRY OF SCIENCE AND HIGHER EDUCATION\\
OF RUSSIAN FEDERATION\\
\setlength{\parskip}{0.3 cm}
FEDERAL STATE BUDGET EDUCATIONAL INSTITUTION\\
OF HIGHER EDUCATION\\
\setlength{\parskip}{0.6 cm}
«NATIONAL RESEARCH UNIVERSITY «MPEI»\\
\setlength{\parskip}{0.8 cm}
Institute of Humanities and Applied Studies\\
Department of Advertising, Public Relations and Linguistics\\
\end{center}

\begin{center}
\setlength{\parskip}{2 cm}
\textbf{QUALIFICATION PAPER}\\
Field of study “Linguistics”
\end{center}

\begin{center}
\setlength{\parskip}{2 cm}
Subject: “COMPARATIVE ANALYSIS OF WAYS TO TRANSLATE COMPUTER TERMINOLOGY FROM ENGLISH INTO RUSSIAN AND SPANISH”
\end{center}

\begin{flushright}
\setlength{\parskip}{4.5 cm}
Student Boyarova, E.A. \\
Academic Supervisor Rodionova, L.U.  
\end{flushright}


\begin{center}
\setlength{\parskip}{2 cm}
Moscow\\
2018
\end{center}
\thispagestyle{empty}

\newpage
\begin{center}
МИНИСТЕРСТВО НАУКИ И ВЫСШЕГО ОБРАЗОВАНИЯ РОССИЙСКОЙ ФЕДЕРАЦИИ\\
\setlength{\parskip}{0.3 cm}
«НАЦИОНАЛЬНЫЙ ИССЛЕДОВАТЕЛЬСКИЙ УНИВЕРСИТЕТ «МЭИ»\\
\setlength{\parskip}{0.6 cm}
Гуманитарно-прикладной институт\\
Кафедра рекламы, связей с общественностью и лингвистики\\
\end{center}

\begin{center}
\setlength{\parskip}{2 cm}
\textbf{ВЫПУСКНАЯ КВАЛИФИКАЦИОННАЯ РАБОТА}\\
по направлению «Лингвистика»
\end{center}

\begin{center}
\setlength{\parskip}{2 cm}
Тема: «\textbf{СОПОСТАВИТЕЛЬНЫЙ АНАЛИЗ СПОСОБОВ ПЕРЕВОДА КОМПЬЮТЕРНОЙ ТЕРМИНОЛОГИИ С АНГЛИЙСКОГО НА РУССКИЙ И ИСПАНСКИЙ ЯЗЫКИ}»\\
\end{center}

\begin{flushright}
\setlength{\parskip}{4 cm}
Студент(ка) группы ГП- 01-14 Боярова Е.А.\\
Научный руководитель - ст. преподаватель/доцент 
Родионова Л.Ю.\\
Допустить к защите…………….зав. кафедрой А. Б. Родин\\
«…»……………….20\dots г.
\end{flushright}


\begin{center}
\setlength{\parskip}{2 cm}
Москва\\
2018
\end{center}
\thispagestyle{empty}

\newpage
\begin{abstract}
   This research is a comparative analysis of techniques used to translate  computer terminology from the English into the Spanish and Russian languages, the material being taken from computer-related texts, entertainment portals and information network resources. In the course of this study, the concept of "computer vocabulary"\  has been investigated, the most frequently employed methods of translating compu\-ter terminology have been studied and difficulties resulting from such translation from English into Russian and Spanish identified. Special emphasis has been laid on an analysis of most efficient translation transformations, and the competences that a translator must have to be successful in  translating computer-themed texts.  
\end{abstract} 


\begin{abstract}
     Данное исследование проводилось на основе сопоставительного анализа перевода компьютерной лексики с английского на русский и испанский языки. Лексика для анализа была взята из текстов на компьютерную тематику, с развлекательных порталов и из сетевых ресурсов. В ходе исследования было изучено понятие компьютерной лексики, также были рассмотрены особенности перевода лингво-культурной адаптации компьютерного термина и трудности, с которыми может столкнуться переводчик при работе с ними. Был проведен анализ переводческих трансформаций, а также были определены качества, которыми должен обладать переводчик для успешного выполнения профессиональной переводческой задачи.  
\end{abstract} 
\thispagestyle{empty}

\newpage
\begin{center}
\tableofcontents 
\end{center}
\thispagestyle{empty}

\newpage
\begin{center}
    \section*{Introduction}\\
\end{center}

This work is devoted to a comparative study of the specifics of the translation of vocabulary related to the field of computer technology. It analyzes the ways of translating computer vocabulary from English into Russian and Spanish. Particular attention is paid to the identification of translation techniques and transformations employed for an adequate translation of computer vocabulary. The concept, origin and specificity of computer terminology have also been considered.\\

The \textbf{relevance} of the chosen topic is due to the increased attention specialists pay to the research of the functioning of the language in the field of computer technologies and the growing role of mass media in the global dissemination of multimedia technologies and expanded access to network databases. At the begin\-ning of the XXI century, the post-industrial civilization entered the era of electro\-nics and computer technology. The Internet revolutionized communication systems. A personal computer has become an integral part of both work and leisure. Compu\-ter skills came to be regarded as attributes of a modern and successful person, and are often a pre-requisite for employment. Great attention is paid to computer literacy in the educational system: classes in schools are equipped with computers, access to the Internet is provided. The development of a new branch of knowledge leads to the development and systematization of special terminology. In this connec\-tion, special attention is paid to research aimed at identifying the specificity of computer terminology and the ways of its translation and adaptation into foreign lingvo-cultural environments.\\

The \textbf{purpose} of this work is to analyze and compare the specifics of the translation of English-language computer vocabulary into Russian and Spanish and compare the methods of its adaptation in these two languages.\\

The objectives of the study are:\\
\begin{itemize}
    \item understanding the concept of computer vocabulary as such;
    \item studying the terminology of the computer field;
    \item analysis of the formation of a computer term;
    \item identification of the specifics and features of the translation of computer vocabulary;
    \item analysis of the concept of abbreviation;
    \item methods and peculiarities of translating abbreviations;
    \item analysis of the reasons for borrowings and calques; 
    \item study of the process of adaptation of borrowings;
    \item analysis of translation transformations used in the process of adapting computer vocabulary to Russian and Spanish linguistic environment;
    \item identification of the qualities that a successful interpreter must possess.\\
\end{itemize}

The \textbf{object} of research is computer vocabulary. The subject of the paper is a comparative study of methods and techniques of translation of computer terms from English into Russian and Spanish.\\

As \textbf{material} of the study we have chosen texts of computer subjects, texts of computer discourse, scientific and technical articles, reports on Internet forums as well as entertainment portals.\\

\textbf{Structure of the work}: the work consists of Introduction, two chapters of the main part – theoretical and practical, Conclusion, a list of Sources and Applications.\\

In the \textbf{Introduction}, the relevance of the study is justified, the goals and tasks of the work are set.\\

In\textbf{ the first chapter} of this work, the concept of "computer vocabulary" has been examined, and the ways of forming a computer term have been investigated. We have carried out a comparative analysis of translation means of the Spanish and Russian languages. The most effective methods of translating computer vocabulary are analyzed and evaluated. The work also analyzes borrowed words, gives reasons for borrowing and also studies, as a separate problem, techniques of translation of abbreviations. This chapter also identifies the competences that an interpreter must have to be an efficient specialist.\\

The \textbf{second chapter}, a practical one, represents a comparative analysis of the translation transformations used in translating terms found in computer-themed texts from English into Russian and Spanish.\\
The following theses have been contended:
\begin{enumerate}
    \item Russian and Spanish are relatively poor (compared to English) in computer terminology, so they have to borrow words from English with all inevitable grammatical, graphical and semantic modifications.
    \item There are different ways to translate computer terminology from English into Russian and Spanish, the most frequent ones being transliteration, calques, semi-calques.
    \item Earlier, Russian computer vocabulary developed exclusively on the basis of the Russian language. But with the ever-going development of computer technologies, the number of new terms dramatically increased, and along with the spread of personal computers, a huge number of English-language vocabulary entered our language.
    \item Spanish computer terminology appeared as a compilation of terms from various related sciences. The main development of computer vocabulary began in the early 1990s, mainly as a result of contacts with the US and English-speaking countries.
    \item When translating computer vocabulary in the scientific texts, an interpreter can face other problems, such as complex grammatical constructions, turns, transformations, which he can avoid only provided he is fully immersed in the computer environment, constantly enhancing his professional competence.
    \item There are several ways to translate abbreviations in Russian, such as transla\-tion with equivalent Russian abbreviation, also frequently used are borrowing and descriptive translation.
    \item English computer abbreviations are not generally translated into Spanish, they are, more often than not, grammatically adapted by assigning the gender article to them.\\
\end{enumerate}

Practical relevance of this diploma paper lies in the possibility of practical use of research findings by both computer users and programmers, as well as students in their translation classes.


\chapter{Theoretical basis for the investigation}
\section{The concept of computer vocabulary and computer term}

The development of the Internet and computer technologies, as well as the develop\-ment of any other sphere of human activity, contributes to the emergence of new lexical units in the language. L. S. Barkhudarov says that practically all areas of human activity are associated with the use of language, and the Internet is no exception [1]. A lot of specialists mentioned what came to be known as a kind of "terminological explosion", which we have witnessed in recent decades. Computer terminology is a relatively young lexical layer, lately it has expanded exponentially, becoming more open to the general public. Today, each user of a personal computer (PC) has a certain lexical stock in the field of computer terminology. The terminology refers to a variety of the national language, a set of lexical units that "denote the concepts of a particular area of knowledge or activity"[14].

By computer vocabulary we mean a kind of special language used by both a professional group of IT specialists and other computer users.\\

According to S.V. Grinev [7], terminology is based on terminological units, such as:
\begin{enumerate}
    \item terms (lexical units of a special language, words or word-combinations used for the exact designation of specific concepts);
    \item terminoids (words for denoting still unsettled, emerging concepts);
    \item preterms (model of terms characterized by semantic fuzziness, distribution in colloquial speech among representatives of a certain professional sphere).
\end{enumerate}
Computer terminology is a part of special (computer) vocabulary. I.L. Komleva calls this vocabulary "computer language", meaning by this definition "a special language formed in the subject area, technologically related to the production of personal computers and software to them" [10, p. 16].

The central concept around which this language is formed is the concept of "computer". The popular notion of "information and communication technologies" \ (ICT) is broader and includes other technologies (television, cellular communica\-tions, etc.). Accordingly, the terminological vocabulary of the computer language will be called "computer terminology".

Russian computer terminology began to develop on the basis of the Russian language. However, as the personal computer use expanded, a huge amount of English-language vocabulary came into our language, and many of the familiar concepts were replaced by borrowed analogues. 

Spanish computer terminology appeared in the middle of the 20th century as a compilation of terms from various related sciences. The main replenishment and development of computer vocabulary began in the early 1990s, mainly as a result of contacts with the US and English-speaking countries and it still continues enriching with new modern terms.

An important feature of the formation of computer terminology of the Spanish language is its origin on the Anglo-American soil, which was expressed in a large number of borrowed words from the English language.

It would be no exaggeration to state that today, the international computer language has a pronounced English-speaking color. The overwhelming majority of modern computer terms are neologisms: \textbf{a processor, a scanner, an interface, a monitor, a modem, etc.}, borrowed from the English language (usually having Greek-Latin roots).

Computer language is characterized by stylistic heterogeneity. It includes litera\-ry vocabulary, represented by general technical terminology, and non-literary: compu\-ter professionalisms and jargon. The vocabulary of each category fulfills its function and has its own area of use. The computer terminology does not differ from other terminology systems. The main purpose of its units is the designation of specific concepts of the computer sphere.

\subsection{Formation and evolution of computer terminology}

The processes of development of computer science and technology are so fast that the Russian and Spanish languages are constantly updated with new computer terms, gradually getting rid of the old ones and often changing the meaning of the already established ones. Alongside with the other ways of forming computer vocabulary, borrowing is the most productive way.

Borrowing of foreign words is one of the ways to develop a modern language. Language always reacts quickly and flexibly to the needs of society. Borrowings are the result of contacts, relationships between peoples and states.

The main reason for borrowing of foreign vocabulary is the absence of the corresponding concept in the cognitive basis of the recipient language.

During the research, it was revealed that all the reasons for the penetration of Anglicisms into the computer terminology of the Spanish and Russian languages can be divided into two types: intra-linguistic and extra-linguistic reasons [19].

Among extra-linguistic reasons for borrowing, most researchers mention: 
\begin{itemize}
    \item increased interest in learning English, associated with the promotion of Anglo-American culture through Internet resources; 
    \item historically determined leadership of the United States in the production and distribution of computer equipment; 
    \item availability of oral and written contacts, provided by cultural and economic cooperation of countries; 
    \item the cultural affinity of languages that has developed over many years and creates favorable conditions for the integration of new terms in the language culture of various social strata of the population of Spain and Russia.\\
\end{itemize}

Among intra-linguistic reasons are:\\

\begin{itemize}
    \item the absence, in the native language, of an equivalent word for a new subject or concept;
    \item tendency to use one borrowed word instead of descriptive phrase;
    \item the need for detailed elaboration of the corresponding meaning, the designa\-tion with the help of a foreign language of some special kind of objects or concepts;
    \item tendency to replenish foreign-language stylistic synonyms.\\
\end{itemize}

Borrowing of words is a natural process of language development. It enriches the language and, if certain reasonable limits are met, does not affect its identity in any way, as new words are sooner or later adapted in accordance with the norms of the language and gradually, as they become assimilated, become an organic part of it.

Borrowed words from the English language, which are embedded in the structure of another language and influence all meaningful language levels, are commonly called Anglicisms, that is the direct or indirect influence of the English language on the phonetic, lexical and syntactic structures of another language [1].

This term is usually understood as various borrowings from the English language: direct borrowing, semantic and word-forming calque, occasional terms and turns.\\

In the Russian and Spanish languages, there are several ways to borrow, that is to create computer vocabulary [21]:\\

\begin{itemize}
    \item Graphical way – is the transmission of a word using the letters of a borrowing language [15], for example:\\
    
    \begin{table}[H]
        \centering
        \begin{tabular}{ | m{3cm} | m{3cm}| m{3cm} | m{3cm} | } 
            UV laser gives the possibility of almost all insulation materials marking without any material damage [38]. &  УФ лазер дает возможность маркировать практически все изоляционные материалы без повреждений защитного слоя. & El láser UV ofrece la posibilidad de marcar casi todos los materiales de aislamiento sin ningún daño material. \\ 
        \end{tabular}
    \end{table}
\end{itemize}

\begin{itemize}
    \item Grammatical way – is the subordination of borrowed word to the rules of the grammar of the borrowing language [15], for example:\\
    
    \begin{table}[H]
        \centering
        \begin{tabular}{ | m{3cm} | m{3cm}| m{3cm} | m{3cm} | } 
            As developing countries industrialize, emissions are likely to increase [38]. &  По мере расширения индустриализации в развивающихся странах эмиссия, вероятно, будет возрастать. & A medida que los países en desarrollo se industrialicen, es probable que aumenten las emisiones. \\ 
        \end{tabular}
    \end{table}
\end{itemize}

\begin{itemize}
    \item Semantic way –  refers to such a process, as a result of which a foreign language enters the system of concepts of the borrowing language [15], for example: \\
    
    \begin{table}[H]
        \centering
        \begin{tabular}{ | m{3cm} | m{3cm}| m{3cm} | m{3cm} | } 
            Google+ is a new social net [38]. &   Google+ - это новая социальная сеть. & Google+ es una nueva red social. \\ 
        \end{tabular}
    \end{table}
\end{itemize}

\subsection{Techniques used for translation of computer terms}

\section{Translation of computer terms in computer texts: challenges and techniques}

\section{The concept of abbreviation}

\section{Methods and peculiarities of translation of abbreviations}

\chapter{Comparative analysis of translation from English into Russian and Spanish}

\begin{thebibliography}
\bibitem{1} Lopez Zurita, P. Economic anglicisms: adaptation to the Spanish linguistic system.
\bibitem{2} Pountain, С. J. Spanish and English in the 21st century.
\bibitem{3} Rodriguez Barrientos, M. Extranjerismos en el lenguaje de la informatica. 
\bibitem{4} Rollason, С. Language Borrowings in a Context of Unequal Systems: Anglicisms in French and Spanish. 
\bibitem{5}	Santiago, R. E. Infotech: English for computer users: Teacher's Book. - fourth edition. - Cambridge: Cambridge Universitety Press, 2008. – 160 р.   
\bibitem{6}	Wallace, L. Meaning and the Structure of Language. – Chicago-London: University of Chicago Press, 1970. – 360 p.
\bibitem{7}	Бархударов, Л.С. Язык и перевод: вопросы общей и частной теории перевода. – М.: Международные отношения, 1975. – 237 с. 
\bibitem{8} Бондарец, О.Э. Категория рода заимствованных слов в русском и испанском языках. 
\bibitem{9} Борисова, Л. И. Лексические особенности англо-русского научно-   технического перевода. – М.: НВИТЕЗАРУС, 2005. – 216 с.
\bibitem{10} Вейзе, А. А. Перевод технической литературы с английского на русский: учеб. пособ. / А. А. Вейзе, Н. Б. Киреев. – Минск, 1997. – 112 с.
\bibitem{11}	Виноградов, В. С. Введение в переводоведение. – М.: Издательство  института общего среднего образования РАО, 2001. – 224 с.
\bibitem{12}	Виноградов, В. С. Лексикология испанского языка Текст. / В. С. Виноградов. М. : Высшая школа, 20036. - 244 с.
\bibitem{13}	Гринев, С. В. Введение в терминоведение. — М., 1993. — 309 с.  
\bibitem{14}	Дорожкина, В. П. Английский язык для математиков. – М.: Изд-во МГУ, 2006. – 411 с.
\bibitem{15}	Колупаева, Е. В. Способы заимствования английских компьютерных терминов в русском языке. 
\bibitem{16}	Комлева, И. Л. Принципы формирования русской компьютерной терминологии: автореф. дис…канд. филолог. наук / И. Л. Григорьева. – Москва: РУДН, 2006.
\bibitem{17}	Кузнецова, Н.А. Динамика функционирования англицизмов в  современном испанском языке. 
\bibitem{18}	Лейчик, М.В. Терминоведение. М.: Либроком, 2009. – 256 с.
\bibitem{19}	Лесниковская, И.В. Экстралингвистические и интралингвистические причины  возникновения ново лексики. 
\bibitem{20}	Миньяр-Белоручев, Р. К. Теория и методы перевода: учеб. пособ. для вузов / Р. К. Миньяр-Белоручев. – М.: Московский Лицей, 1996. – 298 с.
\bibitem{21}	Освоение заимствований и его виды (графическое, грамматическое, семантическое).  
\bibitem{22}	Особенности перевода научно-технических текстов в английском языке. 
\bibitem{23}	Паневина, О. С. Проблема перевода и толкования компьютерной лексики. 
\bibitem{24}	Приемы перевода технической сопроводительной документации.
\bibitem{25}	Рецкер, Я. И. Пособие по переводу с английского на русский. – М.: Наука, 1960. – 343 с.
\bibitem{26}	Тибилова, М.И. Особенности освоения иноязычных аббревиатур.     



\end{thebibliography}
















\end{document}